%!TEX root = ../thesis.tex

\section{まとめ}
%\subsection{Navigation2}
%\subsection{emcl2}
%\section{論文の構成}
\begin{figure}[hbtp]
  \centering
 %\includegraphics[keepaspectratio, scale=0.8]
      %{images/RaspberryPiMouse.png}
 %\caption{Example}
 %\label{Fig:Example}
\end{figure}

\subsection{emcl2 パラメータ調整に関する総括}

本研究では,emcl2 における自己位置推定性能がロボットの走行挙動に
与える影響を明らかにするため,
オドメトリ誤差モデルおよびレーザ尤度モデルに関する
複数のパラメータを変更し,その挙動を比較・評価した.

まず,前進量に比例するオドメトリ誤差を表す
fw\_dev\_per\_fw の実験では,
値を大きくするにつれてロボットの位置推定誤差が累積し,
直進走行時に進行方向から徐々にずれていく挙動が確認された.
特に 0.5 の場合にはゴールに到達できず,
自己位置推定においてオドメトリ誤差の影響が支配的になることが
示された.一方で,0.4 以下ではゴール可能であり,
適切な範囲での設定が重要であることが分かった.

fw\_dev\_per\_rot に関する実験では,
すべての設定値においてゴール到達が可能であった.
値を大きくした場合には位置のずれが確認されたものの,
ナビゲーション全体への致命的な影響は見られず,
回転時の前進誤差は比較的ロバストであることが示唆された.

次に,前進量に比例する回転誤差を表す
rot\_dev\_per\_fw の実験では,
値を大きくしてもロボットはゴールまで到達可能であったが,
直進区間においてロボットの姿勢が徐々にずれていく傾向が確認された.
特に,高い設定値では進行方向が基準値よりも小さくなり,
姿勢推定精度の低下が見られた.

回転量に比例する回転誤差を表す
rot\_dev\_per\_rot の実験では,
値を大きくすることでレーザセンサを重視した自己位置推定となり,
ロボットの挙動が不安定になる傾向が確認された.
特に曲線走行時には,
ゴール付近で停止する事例が観測され,
過度なセンサ重視はナビゲーション性能を低下させる可能性が示された.

レーザ尤度モデルに関する実験では,
laser\_likelihood\_max\_dist を大きく設定すると,
レーザスキャンの評価が必要な場面や,
スキャン対象物が多い環境において,
ロボットが一時的に停止する挙動が確認された.
一方で,0.5 程度までの設定では,
安定した走行が可能であることが分かった.

また,range\_threshold の実験では,
基準値では近距離のスキャンを用いた安定した自己位置推定が行われていたが,
値を大きくすると遠距離のスキャンを用いるため,
位置および姿勢推定にずれが生じる場合があった.
しかし,0.5 程度までであれば,
安定性を保ったままナビゲーションを行うことが可能であった.

以上の結果より,emcl2 における各パラメータは,
オドメトリ情報とレーザセンサ情報の信頼度のバランスを調整する役割を持ち,
設定値によって自己位置推定の特性および
ロボットの走行挙動が大きく変化することが明らかとなった.
そのため,実環境におけるナビゲーションでは,
各パラメータの意味を理解した上で,
環境特性に応じた適切な調整が重要であるといえる.


\subsubsection{etc...}
\newpage