%!TEX root = ../thesis.tex

%\section{まとめ}
%\subsection{Navigation2}
%\subsection{emcl2}
%\section{論文の構成}
%\begin{figure}[hbtp]
  %\centering
 %\includegraphics[keepaspectratio, scale=0.8]
      %{images/RaspberryPiMouse.png}
 %\caption{Example}
 %\label{Fig:Example}
%\end{figure}

%\section{結論}

本研究では,屋外自律移動ロボットにおける安定したナビゲーションの実現を目的として,
Navigation2 を構成する主要なモジュールである
自己位置推定(emcl2),Controller,Planner,Costmap,
および Velocity Smoother の各種パラメータが,
ロボットの走行挙動に与える影響について体系的な実験を行った.

まず,emcl2 に関する実験結果から,
自己位置推定におけるオドメトリ情報とレーザセンサ情報の
信頼度のバランスが,
ロボットの走行安定性に大きく影響することが明らかとなった.
オドメトリ誤差を過度に大きく設定した場合には,
推定誤差が累積しゴール到達が困難となる一方で,
レーザセンサを過度に重視した設定では,
走行中の挙動が不安定になる傾向が確認された.
このことから,
自己位置推定においては,
環境特性に応じた適切なパラメータ調整が不可欠であることが示された.

Controller に関する実験では,
最大速度や加速度に関するパラメータが,
ロボットの走行速度や旋回時の安定性に影響を与えることが確認された.
一方で,
一部のパラメータについては,
単独で値を変更しても走行挙動に大きな変化が現れず,
他のパラメータやシステム全体の制約との相互作用によって,
実際の挙動が決定されていることが示唆された.
この結果は,
Controller パラメータの調整において,
単一パラメータの変更ではなく,
複数パラメータを総合的に考慮する必要性を示している.

Planner に関する実験では,
各パラメータがロボットの直接的な運動挙動よりも,
経路の形状や生成タイミングといった
大域的な経路計画に影響を与えることが確認された.
Planner のパラメータ調整は,
走行の安定性を左右するというよりも,
経路生成の柔軟性や滑らかさを整える目的で
行うべきであると考えられる.

Costmap に関する実験結果からは,
Global Costmap のパラメータが
経路計画全体の安全性や余裕度に影響を与える一方で,
Local Costmap のパラメータ,特に解像度設定が,
ロボットの即時的な挙動や走行安定性に
強く影響することが明らかとなった.
このことから,
屋外環境における自律走行では,
Local Costmap の設定が極めて重要であるといえる.

Velocity Smoother に関する実験では,
各パラメータがゴール到達の可否そのものよりも,
走行の滑らかさや応答性に影響を与えることが確認された.
特に,
速度指令の更新頻度や加速度制限の設定は,
走行の安定性と快適性の両立において
重要な役割を果たすことが示された.

さらに,
Controller と Velocity Smoother における
速度制限パラメータの組み合わせ実験から,
ロボットの実際の走行速度は,
単一のパラメータによって決定されるのではなく,
複数の速度制約条件の相互関係によって
決定されることが明らかとなった.
すべての速度制限を同時に緩和した場合には,
速度は向上するものの経路追従性が低下し,
一方で,
一部の制限のみを適切に調整することで,
安定性を保ったまま速度向上が可能であることが確認された.

以上の結果から,
Navigation2 におけるパラメータ調整は,
各モジュールを独立に考えるのではなく,
Localization,Planning,Control の相互関係を理解した上で
総合的に行うことが重要であると結論づけられる.
本研究で得られた知見は,
屋外自律移動ロボットのナビゲーション性能向上に向けた
実践的なパラメータ調整指針として有用であり,
今後の実環境への適用やさらなる性能改善に
寄与するものと考えられる.


%\subsubsection{etc...}
\newpage