%!TEX root = ../thesis.tex

\section{背景}
%\subsection{Navigation2}
%\subsection{emcl2}
%\begin{figure}[hbtp]
%\begin{flushleft}
  %\centering
 %\includegraphics[keepaspectratio, scale=0.8]
      %{images/RaspberryPiMouse.png}
 %\caption{Example}
 %\label{Fig:Example}
\indent 近年,屋外自律移動ロボットの研究が盛んに行われており,ROS 2 を用いた自律移動システムの実環境への適用が進んでいる.本研究室では,屋外環境における自律移動ロボットの研究を行っており,津田沼チャレンジやつくばチャレンジといった実環境での自律走行実験に参加している.

これらの実験の場では,ロボットが屋外環境を安定して走行するために,高いナビゲーション性能が求められる.Navigation2 を用いた自律移動では,自己位置推定,経路計画,障害物回避などを実現するために,多数のパラメータを適切に調整する必要がある.しかし,私が津田沼チャレンジでロボットの調整作業を行う過程において,パラメータ設定によって意図しない挙動や走行性能の低下が生じることが確認された.
これらの問題は,パラメータの数が多く,それぞれがロボットの挙動に与える影響が分かりにくいことに起因していると考えられる.その結果,問題が発生した際に,どのパラメータが原因であるかを特定することが難しく,調整作業の効率が低下するという課題が生じている.
%そのため,Navigation2 におけるパラメータを変化させた際のロボットの挙動を整理し,両者の関係を明らかにすることは,今後のパラメータ調整の効率化や,問題の挙動の原因究明において重要である.だから、ロボットの挙動に影響があるパラメータの値を1つずつ変化させて実験することで、ロボットの挙動や走行性能を検証する.
そこで本研究では,Navigation2 に含まれる各種パラメータを体系的に変化させ,その際のロボットの走行挙動を実験的に評価・比較する.これにより,パラメータ設定とロボットの挙動との関係を整理し,走行性能低下の要因を明確にするとともに,今後のパラメータ調整における指針を示すことを目的とする.
%\end{flushleft}
%\end{figure}

%\subsubsection{etc...}
\newpage
\section{目的}
%\begin{figure}[hbtp]
  %\centering
 %\includegraphics[keepaspectratio, scale=0.8]
      %{images/RaspberryPiMouse.png}
 %\caption{Example}
 %\label{Fig:Example}
%Navigation2におけるパラメータ調整でのロボットの挙動の変化を調べることを目的とする

本研究の目的は,Navigation2 における各種パラメータを変化させた際に,
自律移動ロボットの走行挙動がどのように変化するかを実験的に評価し,
パラメータ設定がロボットの挙動に与える影響を明らかにすることである.

%\end{figure}

%\subsubsection{etc...}
%\newpage
\section{論文の構成}
\begin{figure}[hbtp]
  %\centering

  本論文の構成を述べる.第1章では、本研究の背景と目的について述べた.第2章では本研究で使用される要素技術について述べる.第3章では調査するパラメータの概要について述べる.
  第4章では実験と考察ついて述べる.
  第5章では本論文の結論を述べる.
  
 %\includegraphics[keepaspectratio, scale=0.8]
      %{images/RaspberryPiMouse.png}
 %\caption{Example}
 %\label{Fig:Example}
\end{figure}

%\subsubsection{etc...}
\newpage