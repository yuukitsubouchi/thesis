%!TEX root = ../thesis.tex

\section{背景}
%\subsection{Navigation2}
%\subsection{emcl2}
%\begin{figure}[hbtp]
%\begin{flushleft}
  %\centering
 %\includegraphics[keepaspectratio, scale=0.8]
      %{images/RaspberryPiMouse.png}
 %\caption{Example}
 %\label{Fig:Example}
\indent 近年,屋外自律移動ロボットの研究が盛んに行われており,ROS 2 を用いた自律移動システムの実環境への適用が進んでいる.本研究室では,屋外環境における自律移動ロボットの研究を行っており,津田沼チャレンジやつくばチャレンジといった実環境での自律走行競技に参加している.

これらの競技では,ロボットが屋外環境を安定して走行するために,高いナビゲーション性能が求められる.Navigation2 を用いた自律移動では,自己位置推定,経路計画,障害物回避などを実現するために,多数のパラメータを適切に調整する必要がある.しかし,ロボットの調整作業を行う過程において,パラメータ設定によって意図しない挙動や走行性能の低下が生じることが確認された.
これらの問題は,パラメータの数が多く,それぞれがロボットの挙動に与える影響が分かりにくいことに起因していると考えられる.
そのため,Navigation2 におけるパラメータを変化させた際のロボットの挙動を整理し,両者の関係を明らかにすることは,今後のパラメータ調整の効率化や,問題の挙動の原因究明において重要である.

%\end{flushleft}
%\end{figure}

%\subsubsection{etc...}
\newpage
\section{目的}
%\begin{figure}[hbtp]
  %\centering
 %\includegraphics[keepaspectratio, scale=0.8]
      %{images/RaspberryPiMouse.png}
 %\caption{Example}
 %\label{Fig:Example}
Navigation2におけるパラメータ調整でのロボットの挙動の変化を調べることを目的とする


%\end{figure}

%\subsubsection{etc...}
%\newpage
\section{論文の構成}
\begin{figure}[hbtp]
  %\centering

  本論文では以下のように構成される
  
  2章では本研究で使用される要素技術

  3章では調査するパラメータの概要

  4章では実験について

  5章では本論文の結論
 %\includegraphics[keepaspectratio, scale=0.8]
      %{images/RaspberryPiMouse.png}
 %\caption{Example}
 %\label{Fig:Example}
\end{figure}

%\subsubsection{etc...}
\newpage