%!TEX root = ../thesis.tex
\section{ROS2}

ROS 2(Robot Operating System 2)\cite{ros2_docs:online} は,
自律移動ロボットをはじめとするロボットシステムを構築するための
ソフトウェアフレームワークである.
本研究では,ロボットの自律移動機能を実現する基盤として ROS 2 を用い,
その上で動作する Navigation2 を使用して走行実験を行った.

ROS 2 では,各機能がノードとして独立して構成されており,
自己位置推定,経路計画,移動制御といった機能が
パラメータ設定を通じて個別に調整可能である.
本研究におけるパラメータ調整は,
ROS 2 におけるこのパラメータ設定機構を用いて実施した.

%ROS 2(Robot Operating System 2)\cite{ros2_docs:online}は,自律移動ロボットをはじめとする様々なロボットシステムを構築するためのミドルウェアである.近年,ロボットは研究用途にとどまらず,幅広い商用分野や実環境での利用が進んでおり,高い信頼性や拡張性が求められている.

%従来の ROS 1 は,モジュラーなフレームワークと豊富なオープンソースコンポーネントによってロボット研究の発展に大きく貢献してきたが,実運用を想定した通信の信頼性やリアルタイム性などの点で課題があった.ROS 2 はこれらの課題を解決するために設計され,分散システムとしての拡張性や信頼性を重視したアーキテクチャを備えている.

%このように ROS 2 は,センシング,経路計画,移動制御,自律機能といったロボットに共通する要素を統合的に扱うことができ,研究から実運用まで幅広い用途に対応可能なロボット開発基盤である.
%ROS 2(Robot Operating System 2)は,自律移動ロボットをはじめとする様々なロボットシステムを構築するためのミドルウェアである.近年,ロボットは研究用途にとどまらず,物流,サービス,インフラ点検など幅広い商用分野へと活用が広がっており,実環境での信頼性や拡張性が強く求められている.このような背景のもと,従来の ROS 1 が担ってきた役割を引き継ぎつつ,実運用を見据えて再設計されたのが ROS 2 である.

%ROS 1 は,モジュラーなフレームワークと豊富なオープンソースコンポーネントを提供することで,ロボット研究の発展を大きく加速させてきた.一方で,通信の信頼性やリアルタイム性,マルチロボット対応,セキュリティといった実運用レベルで求められる機能については,設計段階で十分に考慮されていなかった.そのため,屋外環境や商用システムへの適用には課題が残されていた.

%ROS 2 はこれらの課題を解決するため,アーキテクチャを基礎から見直し,分散システムとしての拡張性や信頼性を重視して設計されている.通信基盤には DDS(Data Distribution Service)を採用し,ノード間通信の信頼性やリアルタイム性の向上を実現している.また,システムの規模や用途に応じた柔軟な構成が可能であり,小規模な研究用ロボットから大規模な商用ロボットシステムまで幅広く対応できる.

%このように,ROS 2 はモジュラー性,スケーラビリティ,信頼性の高いアーキテクチャを備え,センシング,経路計画,移動制御,自律機能といったロボットに共通する要素を統合的に扱うことができるプラットフォームである.そのため,現代のロボットシステムが直面する多様な環境や要求に対応可能な基盤として,研究分野だけでなく実運用の現場においても広く利用されている.
%ロボット技術革命の次の章は,幅広い商用用途にロボットが導入されることで既に始まっている.無数のアプリケーションや環境においても,ロボットが共通して必要とするコンポーネントが存在する――モジュラー性,スケーラビリティ,信頼性の高いアーキテクチャ,センシング,経路計画,移動性,自律性である.

%ロボットオペレーティングシステム(ROS)は前の章において重要な役割を果たし,モジュラーなフレームワークと自由に利用可能なコンポーネントによってロボット研究の加速を実証してきた.しかしながら,ROS 1 は多くの実運用レベルの機能やアルゴリズムを念頭に設計されていなかった.ROS 2 とその関連プロジェクトは,現代のロボットシステムがあらゆる規模や新しい応用領域で直面する課題に対処できるよう,基礎から再設計されている.
\section{Navigation2}
Navigation2(Nav2)\cite{nav2_docs:online}は,自動運転車向けに開発されたナビゲーション技術を,移動ロボットおよび地上ロボット向けに導入・最適化・再構築した,ROS におけるナビゲーションスタックの後継フレームワークである.Nav2 を用いることで,移動ロボットは複雑な環境内を自律的に移動し, さまざまな運動学モデルを持つロボットに対してユーザー定義のタスクを実行することが可能となる.

Nav2 は,単にロボットを地点 A から地点 B へ移動させるだけでなく,中間地点を含む経路の実行や,物体追跡,領域全体を網羅するカバレッジナビゲーションなど,多様なナビゲーションタスクを表現・実行できる.また,知覚,経路計画,制御,自己位置推定,可視化といった,自律移動に必要な機能を統合的に提供することで,高い信頼性を持つ自律移動システムの構築を可能にしている.

Nav2 では,センサ情報や環境モデルを基に動的な経路計画を行い,障害物を回避しながらモータの速度指令を生成することで,状況に応じた柔軟な移動を実現する.さらに,複数の独立したモジュラー構成のサーバをビヘイビアツリー(Behavior Tree)によって統合・制御することで,経路計算や動作制御などの各機能を組み合わせた柔軟なナビゲーション動作を実現している.この構成により,ロボットは複雑かつ多様な自律移動タスクを実行することが可能となる.
\figref{Fig:Structure of Navigation2}にNavigation2の構造を示す
\begin{figure}[hbtp]
     \centering
\includegraphics[keepaspectratio, scale=0.3]
     {images/nav2_architecture.png}
     \caption{Structure of Navigation2}
     \label{Fig:Structure of Navigation2}
\end{figure}
%\subsection{Navigation2}
%\subsection{emcl2}
%\newpage
\section{emcl2}

emcl2(Extended Monte Carlo Localization 2)\cite{emcl2/ryuichiueda}は,上田隆一らによって提案された自己位置推定手法である Extended Monte Carlo Localization(emcl)を,ROS 2 環境で利用可能としたソフトウェアである.

emcl は,パーティクルフィルタに基づく Monte Carlo Localization(MCL)を拡張した手法であり,事前に与えられた地図とセンサ情報を用いてロボットの自己位置を推定する.特に,外乱や自己位置推定の破綻に対するロバスト性を高めるための拡張リセット機構を備えている点に特徴がある.

emcl2 は,この emcl のアルゴリズムを ROS 2 向けに実装したものであり,AMCL と同様に,オドメトリ情報による状態遷移モデルと,LiDAR などの距離センサに基づく観測モデルを用いて,各パーティクルの尤度評価と再サンプリングを行うことで,ロボットの自己位置を逐次的に推定する.

本研究では,ROS 2 Navigation2 環境における自己位置推定手法として emcl2 を用い,パラメータ設定の違いがロボットの挙動およびナビゲーション性能に与える影響を評価する.
%emcl2(Extended Monte Carlo Localization 2)\cite{emcl2/ryuichiueda}は,ROS2 環境において利用可能な自己位置推定手法の一つであり,パーティクルフィルタに基づく Monte Carlo Localization(MCL)を拡張したローカライゼーションアルゴリズムである.emcl2 は,従来の AMCL と同様に,事前に与えられた地図とセンサ情報を用いてロボットの自己位置を推定するが,より高いロバスト性と拡張性を持つ点に特徴がある.

%emcl2 では,複数のパーティクルを用いてロボットの位置姿勢の確率分布を表現し,オドメトリ情報による状態遷移と,LiDAR などの距離センサを用いた観測モデルに基づいて,各パーティクルの尤度を評価する.これらの尤度に基づいて再サンプリングを行うことで,ロボットの自己位置を逐次的に推定する.また,emcl2 はセンサの外乱や環境変化に対しても安定した推定が行えるように設計されている.
%\section{論文の構成}
\begin{figure}[hbtp]
  \centering
 %\includegraphics[keepaspectratio, scale=0.8]
      %{images/RaspberryPiMouse.png}
 %\caption{Example}
 %\label{Fig:Example}
\end{figure}

%\subsubsection{etc...}
%\newpage