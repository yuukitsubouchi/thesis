%!TEX root = ../thesis.tex
\section{ROS2}
ロボット技術革命の次の章は,幅広い商用用途にロボットが導入されることで既に始まっている.無数のアプリケーションや環境においても,ロボットが共通して必要とするコンポーネントが存在する――モジュラー性,スケーラビリティ,信頼性の高いアーキテクチャ,センシング,経路計画,移動性,自律性である.

ロボットオペレーティングシステム(ROS)は前の章において重要な役割を果たし,モジュラーなフレームワークと自由に利用可能なコンポーネントによってロボット研究の加速を実証してきた.しかしながら,ROS 1 は多くの実運用レベルの機能やアルゴリズムを念頭に設計されていなかった.ROS 2 とその関連プロジェクトは,現代のロボットシステムがあらゆる規模や新しい応用領域で直面する課題に対処できるよう,基礎から再設計されている.
\section{Navigation2}
%\subsection{Navigation2}
%\subsection{emcl2}
\section{emcl2}
%\section{論文の構成}
\begin{figure}[hbtp]
  \centering
 %\includegraphics[keepaspectratio, scale=0.8]
      %{images/RaspberryPiMouse.png}
 %\caption{Example}
 %\label{Fig:Example}
\end{figure}

%\subsubsection{etc...}
\newpage