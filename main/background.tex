%!TEX root = ../thesis.tex

\section{調査するパラメータ}
\begin{figure}[hbtp]
  \centering
 %\includegraphics[keepaspectratio, scale=0.8]
      %{images/RaspberryPiMouse.png}
 %\caption{Example}
 %\label{Fig:Example}
\end{figure}
%次に調査するパラメータを紹介する。下の表に示しているのが今回研究に使われたパラメータである.
%これらのパラメータは、ロボットの走行や追従、回避などに関連しているものを選んでいる.

本研究では,ROS 2 Navigation2 における各種パラメータの設定が,自律移動中のロボットの挙動に与える影響を明らかにすることを目的とする.

Navigation2 は,自己位置推定,経路計画,障害物回避,移動制御といった複数の機能から構成されており,それぞれに多数のパラメータが存在する.これらのパラメータは相互に影響し合うため,設定値の違いによって走行速度や軌道の安定性,さらにはゴール到達可否といった挙動に差が生じる.

本章では,実験において調査対象としたパラメータの概要を示し,次節以降で各パラメータの役割および設定内容について詳しく説明する.


%\begin{table}[htbp]
  %\centering
  \begin{table}[htbp]
  \centering
  \caption{Parameter List in emcl2}
  \label{tab:emcl2_params}
  \begin{tabular}{p{0.8\textwidth}}
    \toprule
    Parameter Name \\
    \midrule
    num\_particles \\
    odom\_fw\_dev\_per\_fw \\
    odom\_fw\_dev\_per\_rot \\
    odom\_rot\_dev\_per\_fw \\
    odom\_rot\_dev\_per\_rot \\
    laser\_likelihood\_max\_dist \\
    range\_threshold \\
    \bottomrule
  \end{tabular}
\end{table}

\begin{table}[htbp]
  \centering
  \caption{Parameter List of the Nav2 Controller}
  \label{tab:controller_params}
  \begin{tabular}{p{0.8\textwidth}}
    \toprule
    Parameter Name \\
    \midrule
    max\_vel\_x \\
    max\_vel\_theta \\
    max\_speed\_xy \\
    min\_theta\_velocity\_threshold \\
    acc\_lim\_x \\
    acc\_lim\_theta \\
    linear\_granularity \\
    angular\_granularity \\
    xy\_goal\_tolerance \\
    \bottomrule
  \end{tabular}
\end{table}

\begin{table}[H]
  \centering
  \caption{Parameters of the Nav2 Costmap}
  \label{tab:costmap_params}
  \begin{tabular}{p{0.8\textwidth}}
    \toprule
    Parameter Name \\
    \midrule
    global\_resolution \\
    global\_cost\_scaling\_factor \\
    global\_inflation\_radius \\
    local\_resolution \\
    local\_cost\_scaling\_factor \\
    local\_inflation\_radius \\
    \bottomrule
  \end{tabular}
\end{table}

\begin{table}[H]
  \centering
  \caption{Parameters of the Nav2 Velocity Smoother}
  \label{tab:smoother_params}
  \begin{tabular}{p{0.8\textwidth}}
    \toprule
    Parameter Name \\
    \midrule
    smoothing\_frequency \\
    max\_velocity \\
    max\_accel\_x \\
    max\_accel\_theta \\
    \bottomrule
  \end{tabular}
\end{table}

\begin{table}[H]
  \centering
  \caption{Parameters of the Nav2 Planner and Goal Checker}
  \label{tab:planner_params}
  \begin{tabular}{p{0.8\textwidth}}
    \toprule
    Parameter Name \\
    \midrule
    %linear\_granularity \\
    %angular\_granularity \\
    %xy\_goal\_tolerance \\
    trans\_stopped\_velocity \\
    planner\_tolerance \\
    \bottomrule
  \end{tabular}
\end{table}



%\begin{table}[htbp]
%\begin{table}[htbp]
  %\centering
  
  %\end{tabular}
 %\end{table}


%\caption{パラメータ一覧}
%\begin{tabular}{cc}
%項目 & 値 \\
%\end{tabular}
%\end{table}

%\subsubsection{etc...}
\newpage

\section{パラメータの詳細}
\subsection{emcl2}

\subsubsection{num\_particles}
本パラメータは,自己位置推定に用いる パーティクルの数を設定するものである.
パーティクル数を増加させることで,自己位置分布をより詳細に表現でき,
推定精度やロバスト性の向上が期待できる.
一方で,計算負荷が増加するため,リアルタイム性とのトレードオフが存在する.

\subsubsection{odom\_fw\_dev\_per\_fw}
前進移動時における前進方向のオドメトリ誤差を表すパラメータである.
値を大きくするとオドメトリを信頼しにくくなり,パーティクルの分散が大きくなる.

\subsubsection{odom\_fw\_dev\_per\_rot}
回転動作時における前進方向のオドメトリ誤差を表すパラメータである.
値を大きくすると,回転時の前進誤差を大きく見積もるため,自己位置推定の不確実性が増加する.
回転動作の多い環境では,推定安定性に影響を与える.

\subsubsection{odom\_rot\_dev\_per\_fw}
前進移動量に対する回転方向のオドメトリ誤差を表すパラメータである.
値を大きく設定すると,直進時の姿勢角誤差を大きく考慮するようになる.
これにより,直進走行中の姿勢推定のばらつきが増加する.

\subsubsection{odom\_rot\_dev\_per\_rot}
回転動作時における回転方向のオドメトリ誤差を表す.
値を大きくすると回転量の信頼度が低下し,姿勢角の推定誤差が大きくなる.
旋回動作の安定性に影響を与える重要なパラメータである.

\subsubsection{laser\_likelihood\_max\_dist}
レーザスキャンを用いた尤度計算において考慮する最大距離を設定するパラメータである.
値を大きくすると遠方の障害物まで尤度計算に含まれるが,計算誤差の影響を受けやすくなる.
屋外環境では環境特性に応じた調整が必要である.

\subsubsection{range\_threshold}
レーザスキャンデータの有効距離の上限を設定するパラメータである.
この値を超える測距データは無効として扱われる.
外乱やノイズの多い環境では,自己位置推定の安定化に寄与する.

\subsection{Navigation2}
\subsubsection{controller}

\subsubsection{max\_vel\_x}
ロボットの前進方向の最大速度を設定するパラメータである.
値を大きくすると高速移動が可能となるが,制御の不安定化や振動が生じやすくなる.
安全性と走行性能のバランスが重要となる.

\subsubsection{max\_vel\_theta}
ロボットの最大回転速度を設定するパラメータである.
値を大きくすると旋回が速くなるが,急激な姿勢変化が発生する可能性がある.
狭い環境では慎重な調整が必要である.

\subsubsection{max\_speed\_xy}
平面移動における最大速度を制限するパラメータである.
前進速度と横方向速度を含めた総合的な移動速度の上限を定める.
移動の滑らかさに影響を与える.

\subsubsection{min\_theta\_velocity\_threshold}
回転速度が停止と判定される最小閾値を表すパラメータである.
値を大きくすると微小な回転が無視されやすくなり,姿勢制御が粗くなる.
ゴール付近での姿勢安定性に影響を与える.

\subsubsection{acc\_lim\_x}
前進方向の加速度制限を設定するパラメータである.
値を小さくすると加減速が緩やかになり,安定した走行が可能となる.
一方で応答性は低下する.

\subsubsection{acc\_lim\_theta}
回転方向の加速度制限を設定するパラメータである.
急激な旋回動作を抑制し,滑らかな姿勢制御を実現するために用いられる.

\subsubsection{linear\_granularity}
経路生成時の直線方向の分割間隔を設定するパラメータである.
値を小さくすると経路が滑らかになるが,計算量が増加する.

\subsubsection{angular\_granularity}
経路生成時の角度方向の分割間隔を設定する.
旋回精度に影響を与える.

\subsubsection{xy\_goal\_tolerance}
ロボットがゴールに到達したと判定される位置誤差の許容範囲を表す.
値を大きくするとゴール判定が緩くなる.

\subsubsection{costmap}
\subsubsection{global\_resolution}
Global Costmap の解像度を設定するパラメータである.
値を小さくすると詳細な地図表現が可能となるが,計算負荷が増加する.

\subsubsection{global\_cost\_scaling\_factor}
障害物からの距離に応じたコストの減衰率を設定するパラメータである.
値を大きくすると障害物付近のコストが急激に低下する.

\subsubsection{global\_inflation\_radius}
障害物の周囲をどの程度膨張させるかを設定するパラメータである.
安全距離の確保に直接影響する.

\subsubsection{local\_resolution}
Local Costmap の解像度を設定するパラメータである.
局所的な障害物回避性能に影響を与える.

\subsubsection{local\_cost\_scaling\_factor}
Local Costmap における障害物コストの減衰率を設定する.
回避挙動の敏感さに影響する.

\subsubsection{local\_inflation\_radius}
Local Costmap における障害物の膨張半径を設定する.
障害物回避時の安全性を左右する.

\subsubsection{Velocity Smoother}
\subsubsection{smoothing\_frequency}
速度平滑化処理の更新周波数を設定するパラメータである.
値を大きくすると滑らかな速度制御が可能となる.

\subsubsection{max\_velocity\_x}
max\_velocityの第1要素は,ロボットの 前後方向(x方向)の最大並進速度を表す.
この値を超える速度指令は制限されるため,ロボットの最高走行速度が規定される.
値を大きくすると移動速度は向上するが,停止距離や制御安定性への影響が大きくなる.

\subsubsection{max\_velocity\_theta}
max\_velocity の第3要素は,ロボットの 回転方向(θ方向)の最大角速度を表す.
旋回時の最大回転速度を制限し,急激な回転動作を防ぐ役割を持つ.
値を大きくすると旋回応答は向上するが,姿勢の安定性が低下する可能性がある.

\subsubsection{max\_accel\_x}
max\_accelの第1要素は,並進方向の最大加速度を制限するパラメータである.
走行時の急発進を抑制する.

\subsubsection{max\_accel\_theta}
max\_accel の第3要素は,回転方向の最大加速度を制限するパラメータである.
旋回動作の安定性向上に寄与する.

\subsubsection{planner/ゴール判定}
%\subsubsection{linear\_granularity}
%経路生成時の直線方向の分割間隔を設定するパラメータである.
%値を小さくすると経路が滑らかになるが,計算量が増加する.

%\subsubsection{angular\_granularity}
%経路生成時の角度方向の分割間隔を設定する.
%旋回精度に影響を与える.

%\subsubsection{xy\_goal\_tolerance}
%ロボットがゴールに到達したと判定される位置誤差の許容範囲を表す.
%値を大きくするとゴール判定が緩くなる.

\subsubsection{trans\_stopped\_velocity}
並進速度が停止と判定される閾値を設定するパラメータである.
ゴール付近での停止判定に影響を与える.

\subsubsection{planner\_tolerance}
経路計画時に許容されるゴール周辺の誤差範囲を設定するパラメータである.
局所的な計画失敗を防ぐ役割を持つ.

\begin{figure}[hbtp]
  \centering
 %\includegraphics[keepaspectratio, scale=0.8]
      %{images/RaspberryPiMouse.png}
 %\caption{Example}
 %\label{Fig:Example}
\end{figure}

%\subsubsection{etc...}
\newpage