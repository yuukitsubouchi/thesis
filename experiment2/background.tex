%!TEX root = ../thesis.tex

\section{実験概要}
これまでの Controller および Velocity Smoother に関する実験において,
各パラメータの上限値を増加させた場合でも,
ロボットの並進速度が一定値以上に増加しない挙動が確認された.
特に,Velocity Smoother における
\texttt{max\_velocity} や,
Controller における
\texttt{max\_vel\_x},
\texttt{max\_speed\_xy}
の値を大きく設定した場合でも,
実際の走行速度は基準値付近に制限されていた.

\begin{figure}[H]
  \centering
 \includegraphics[keepaspectratio, scale=0.6]
      {images/mvx1.0sokudo.png}
 \caption{Robot speed with max\_vel\_x = 1.0
}
 \label{Fig:Robot speed with max_vel_x = 1.0}
\end{figure}

\begin{figure}[H]
  \centering
 \includegraphics[keepaspectratio, scale=0.6]
      {images/msxy1.5sokudo.png}
 \caption{Robot speed with max\_speed\_xy = 1.5
}
 \label{Fig:Robot speed with max_speed_xy = 1.5}
\end{figure}

\begin{figure}[H]
  \centering
 \includegraphics[keepaspectratio, scale=0.6]
      {images/mvx1.2sokudo.png}
 \caption{Robot speed with max\_velocity\_x = 1.2
}
 \label{Fig:Robot speed with max_velocity_x = 1.2}
\end{figure}

この結果から,Navigation2 におけるロボットの並進速度は,
単一のパラメータによって決定されるのではなく,
Controller と Velocity Smoother の複数の速度制限パラメータが
相互に影響し合うことで制約されている可能性が示唆された.

そこで,速度制限に関与するパラメータ間の関係性を明らかにするため,
Controller の \texttt{max\_vel\_x} および \texttt{max\_speed\_xy},
ならびに Velocity Smoother の
\texttt{max\_velocity} を対象として,
それぞれの値を変化させた際のロボットの挙動を比較・評価する
追加実験を行った.

\section{実験方法}
本実験では,
Controller の \texttt{max\_vel\_x} を 1.5 に固定し,
\texttt{max\_speed\_xy} および
Velocity Smoother の速度制限パラメータ\texttt{max\_velocity\_x}を段階的に変更することで,
実際の走行速度およびロボットの挙動への影響を調査した.

Controller および Velocity Smoother における速度制限パラメータの関係性を明らかにするため,
Controller の \texttt{max\_vel\_x} を 1.5 に固定し,
\texttt{max\_speed\_xy} および
Velocity Smoother の \texttt{max\_velocity}(x成分)を
それぞれ 1.5 に設定した組み合わせ実験を行った。
本実験では,以下の 3 通りの組み合わせについて評価を行った.

\begin{itemize}
  \item \texttt{max\_vel\_x = 1.5} と \texttt{max\_speed\_xy = 1.5}
  \item \texttt{max\_vel\_x = 1.5} と \texttt{max\_velocity\_x = 1.5}
  \item \texttt{max\_vel\_x = 1.5},\texttt{max\_velocity\_x = 1.5},
        \texttt{max\_speed\_xy = 1.5}
\end{itemize}


\section{実験結果}
\begin{figure}[H]
  \centering
 \includegraphics[keepaspectratio, scale=0.6]
      {images/mvxmsxy1.5.png}
 \caption{Robot speed with max\_speed\_xy = 1.5
}
 \label{Fig:Robot speed with max_speed_xy = 1.5}
\end{figure}

\begin{figure}[H]
  \centering
 \includegraphics[keepaspectratio, scale=0.6]
      {images/mvxmvx1.5.png}
 \caption{Robot speed with max\_velocity\_x = 1.5
}
 \label{Fig:Robot speed with max_velocity_x = 1.5}
\end{figure}

\begin{figure}[H]
  \centering
 \includegraphics[keepaspectratio, scale=0.6]
      {images/sokudoall1.5.png}
 \caption{Robot speed with max\_velocity\_x = 1.5 and max\_speed\_xy = 1.5
}
 \label{Fig:Robot speed with max_velocity_x = 1.5 and max_speed_xy = 1.5}
\end{figure}

次にロボットの挙動は下のグラフに示す.\texttt{max\_vel\_x}の値は1.5とする.

% \begin{figure}[H]
%   \centering
%  \includegraphics[keepaspectratio, scale=0.6]
%       {images/s1muki.png}
%  \caption{Robot orientation with max\_speed\_xy = 1.5}
%  \label{Fig:Robot orientation with max_speed_xy = 1.5 }
% \end{figure}
\begin{figure}[H]
  \centering
 \includegraphics[keepaspectratio, scale=0.6]
      {images/s1iti2.png}
 \caption{Trajectory of the robot with max\_speed\_xy = 1.5
}
 \label{Fig:Robot position with max_speed_xy = 1.5}
\end{figure}

% \begin{figure}[H]
%   \centering
%  \includegraphics[keepaspectratio, scale=0.6]
%       {images/s2muki.png}
%  \caption{Robot orientation with max\_velocity\_x = 1.5}
%  \label{Fig:Robot orientation with max_velocity_x = 1.5 }
% \end{figure}
\begin{figure}[H]
  \centering
 \includegraphics[keepaspectratio, scale=0.6]
      {images/s2iti2.png}
 \caption{Trajectory of the robot with max\_velocity\_x = 1.5
}
 \label{Fig:Robot position with max_velocity_x = 1.5}
\end{figure}

% \begin{figure}[H]
%   \centering
%  \includegraphics[keepaspectratio, scale=0.6]
%       {images/s3muki.png}
%  \caption{Robot orientation with max\_velocity\_x = 1.5 and max\_speed\_xy = 1.5}
%  \label{Fig:Robot orientation with max_velocity_x = 1.5 and max_speed_xy = 1.5}
% \end{figure}
\begin{figure}[H]
  \centering
 \includegraphics[keepaspectratio, scale=0.6]
      {images/s3iti2.png}
 \caption{Trajectory of the robot with max\_velocity\_x = 1.5 and max\_speed\_xy = 1.5
}
 \label{Fig:Robot position with max_velocity_x = 1.5 and max_speed_xy = 1.5}
\end{figure}


\begin{table}[H]
  \centering
  \caption{Experimental Results for Speed Limitation Parameter Combinations}
  \label{tab:speed_param_combination}
  \begin{tabular}{|l|c|l|}
    \hline
    Increased parameters & Speed change & Goal achievement \\ \hline
    All three parameters &
    Significant increase &
    Unstable, goal not achieved \\ \hline
    max\_vel\_x, max\_velocity\_x &
    Moderate increase &
    Stable, goal achieved \\ \hline
    max\_vel\_x, max\_speed\_xy &
    No increase &
    Stable, goal achieved \\ \hline
  \end{tabular}
\end{table}

%\subsection{速度制限パラメータの組み合わせによる影響}

% Controller および Velocity Smoother における速度制限パラメータの関係性を明らかにするため,
% Controller の \texttt{max\_vel\_x} を 1.5 に固定し,
% \texttt{max\_speed\_xy} および
% Velocity Smoother の \texttt{max\_velocity}(x成分)を
% それぞれ 1.5 に設定した組み合わせ実験を行った。
% 本実験では,以下の 3 通りの組み合わせについて評価を行った.

% \begin{itemize}
%   \item \texttt{max\_vel\_x = 1.5} と \texttt{max\_speed\_xy = 1.5}
%   \item \texttt{max\_vel\_x = 1.5} と \texttt{max\_velocity\_x = 1.5}
%   \item \texttt{max\_vel\_x = 1.5},\texttt{max\_velocity\_x = 1.5},
%         \texttt{max\_speed\_xy = 1.5}
% \end{itemize}

その結果,
3 つすべてのパラメータを 1.5 に設定した場合,
最も高い並進速度が確認されたが,
走行中に経路から外れ,
コース脇の段差方向へ進行したため,
ゴールに到達することはできなかった.
このことから,
速度制限を同時に緩和しすぎると,
経路追従性や安定性が低下することが確認された.

一方,
\texttt{max\_vel\_x = 1.5} と
\texttt{max\_velocity\_x = 1.5} の組み合わせでは,
速度が基準値である 0.6 を超えて明確に上昇したにもかかわらず,
ロボットは安定した挙動を保ち,
ゴールまで到達することができた.
この結果は,
Controller と Velocity Smoother の速度上限を適切に一致させることで,
速度向上とナビゲーションの安定性を両立できる可能性を示している.

また,
\texttt{max\_vel\_x = 1.5} と
\texttt{max\_speed\_xy = 1.5} の組み合わせでは,
走行中の速度は基準値である 0.6 以上には増加せず,
他のパラメータによる制限が依然として有効であることが確認された.
この条件では,
最初の曲がり角付近で一時的に停止する挙動が見られたものの,
最終的にはゴールに到達しており,
ナビゲーション自体は安定していた.

以上の結果から,
Navigation2 における実際の走行速度は,
単一の速度制限パラメータではなく,
Controller と Velocity Smoother にまたがる複数の制約条件の
組み合わせによって決定されることが明らかとなった.
特に,
すべての制限を同時に緩和すると速度は向上するが,
経路追従性能が低下する一方で,
一部の制限のみを調整することで,
安定性を保ったまま速度向上が可能であることが示唆された.
